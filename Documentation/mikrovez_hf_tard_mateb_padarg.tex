\documentclass[12pt]{extarticle}

\usepackage[utf8]{inputenc}
\usepackage[hungarian=nohyphenation]{hyphsubst}
\usepackage[hungarian]{babel}
\usepackage{graphicx}
\usepackage{subcaption}
\usepackage{fancyhdr}
\usepackage{caption}
\usepackage{url}
\usepackage{siunitx}
\usepackage{amsmath}
\usepackage{pdfpages}

\title{\huge Mikrovezérlők projekt feladat\\[10pt]
	\Large LED-sor vezérlés}
\date{\today}
\author{Máté Bálint, Pádár Gergely, Tar Dániel }


\begin{document}
	
	\begin{titlepage}
		\lhead{\includegraphics[height=2cm]{logo_mogi.png}}
		\rhead{\large{\textbf{Biomechatronika projekt}}\\
			\large{BMEGEMIAMBP}}
		
		\maketitle
		\pagenumbering{gobble}
		\thispagestyle{fancy}
		
		\begin{figure}
			\begin{center}
				\includegraphics[height=2cm]{logo_bme_kicsi.eps}
			\end{center}
		\end{figure}
		
	\end{titlepage}
	
	
	\newpage
	\pagenumbering{gobble}
	\tableofcontents
	
	
	\newpage
	\pagenumbering{arabic}
	
	
	\section{Feladat ismertetése}
	
	LED-sor színének beállítása mikrokontrollerrel. Majd vezeték nélküli vezérlés megoldása wi-fi modul segítségével. Továbbá felhasználóbarát Android alkalmazással a LED-ek színének beállítása egy egyszerű GUI-n.
	
	\section{Megoldás részletezése}
	
	Az android alkalmazásnál beállítható a vezérelendő eszköz IP címe. Ezután UDP csomagokat küldünk a céleszközre RGBM felosztásban. A wi-fi modul a beérkező csomagokat fogadja, majd UART-on keresztül továbbküldi a bájtok számát, valamint a beérkezett értékeket a mikrokontrollernek. A mikrokontroller ezt lementi egy adott memória területre. Végül a fő ciklusban a LED-sor által értelmezhető formátumban küldi tovább. 
	
	\section{Felhasznált eszközök}
	
	A szükséges eszközöket eBay-ről szereztük be. A LED-sor kb. 3500 Ft, a mikrovezérlő kb. 650 Ft, a wifi modul kb. 600 Ft, a DC/DC konverter pedig kb. 400 forintba került. Tehát az egész projekthez szükséges elektronika 5500 Ft-ból beszerezhető.
	
	\subsection{LED-sor}
	
	A LED-sor egy szegmense 3 db RGB LED-ből, valamint egy  WS2811 típusú IC-ből áll. Az 5 méter hosszú soron 50 db ilyen szegmens helyezkedik el.  
	
	\subsection{Mikrovezérlő}
	
	A feladathoz az egyik legelterjedtebb, egy STM32F103C8 típusú mikrokontrollert használtunk. 
	
	\subsection{Wi-fi modul}
	
	Ide egy ESP8266 számú modult szeretünk be.
	
	\subsection{DC/DC konverter}
	
	
	
	
	
	
	
	
	
	
	
	
	
	
	
	
	
\end{document}